\documentclass[]{article}
\pagenumbering{gobble}
\usepackage[a3paper]{geometry}
\usepackage{amsmath, amssymb, amscd, amsthm, amsfonts}
\usepackage{graphicx}
\usepackage{tcolorbox}
\usepackage{hyperref}

\title{TD5: IoT Protocols}
\author{Gabriel PEREIRA DE CARVALHO}
\date{Last modification: \today}

\begin{document}
	
	\maketitle
	
	\section{Battery Life Time in LoRa}
	
	\subsection*{Question 1}
	
	Given the bandwidth $B$ and the spreading factor $\mathrm{SF}$ (number of bits per symbol) we have
	
	\begin{align}
		\begin{cases}
			N &= 2^{SF} = 4096\\
			T_{\text{symbol}} &= \frac{N}{B} = 3.28 \cdot 10^{-2} \mathrm{s}
		\end{cases}
	\end{align}
	
	We compute number of symbols in the transmitted packet $n$,
	\begin{equation}
		n = q + \frac{1}{SF}\left(\frac{8p}{r} + o\right) = 81.5 \approx 82 \text{ symbols}
	\end{equation}
	
	now we can compute the packet transmission time
	\begin{equation}
		T_{tx} = n \cdot T_{\text{symbol}} = 2.68 \mathrm{s}
	\end{equation}
	
	\subsection*{Question 2}
	
	Both the receive windows needs to be long enough to detect a preamble, we have
	
	\begin{equation}
		T_{window}^A = q \cdot T_{\text{symbol}} = 0.393 \mathrm{s}
	\end{equation}
	
	now to compute total receive time, we add the two window durations
	
	\begin{equation}
		T_{rx}^A = 2T_{window}^A = 0.786 \mathrm{s}
	\end{equation}
	
	\subsection*{Question 3}
	
	We calculate the average current
	\begin{equation}
		I_{avg} = \frac{I_{tx} \cdot T_{tx} + I_{rx} \cdot T_{rx}^A + I_{\text{sleep}} \cdot T_{\text{sleep}}}{T_{\text{transmission}}} = 47.8 \mathrm{\mu A}
	\end{equation}
	
	and now we compute the battery life
	\begin{equation}
		L = \frac{C}{I_{avg}} = 52256.3 \text{ hours} = 5.96 \text{ years}
	\end{equation}
	
	\subsection*{Question 4}
	
	\subsubsection*{Beacon duration}
	
	Let's compute the duration of a beacon symbol
	
	\begin{equation}
		T_{\text{symbol}}^{\text{beacon}} = \frac{2^{9}}{125\text{kHz}} = 4.1\text{ms}
	\end{equation}
	
	Now let's compute the number of symbols in the beacon. The 17 bytes in the payload with a coding rate of $\frac{4}{5}$ and a spreading factor of 9 bits per symbol are equivalent to $\frac{17 \cdot 8}{\frac{4}{5} \cdot 9} = 19$ symbols. We have
	
	\begin{equation}
		N^{\text{beacon}} = 29\text{ symbols}
	\end{equation}
	
	The beacon duration is 
	
	\begin{equation}
		T^{\text{beacon}} = N^{\text{beacon}} \cdot T_{\text{symbol}}^{\text{beacon}} = 0.12\text{s}
	\end{equation}
	
	\subsubsection*{Ping slot duration}
	
	Let's compute the duration of a symbol in the ping slot
	
	\begin{equation}
		T_{\text{symbol}}^{\text{ping}} = \frac{2^{12}}{125\text{kHz}} = 32.8\text{ms}
	\end{equation}
	
	We know the number of symbols in the ping slot is $N^{\text{ping}} = 12$ symbols.
	
	The ping slot duration is
	
	\begin{equation}
		T^{\text{ping}} = N^{\text{ping}} \cdot T_{\text{symbol}}^{\text{ping}} = 0.39\text{s}
	\end{equation}
	
	\subsubsection*{$T_{rx}^B$ calculation}
	
	The average packet reception time of B is a weighted average of the reception times of beacons and ping slots. Using $T = 2\text{ hours} = 7200\text{ s}$ We have
	
	\begin{equation}
		T_{rx}^B = T_{rx}^A + \frac{T \cdot T_{\text{beacon}}}{t_b} + \frac{T \cdot T_{\text{ping}}}{t_p} = 2838.6 \text{ s}
	\end{equation}
	
	\subsection*{Question 5}
	
	We calculate the average current
	\begin{equation}
		I_{avg}^B = \frac{I_{tx} \cdot T_{tx} + I_{rx} \cdot T_{rx}^B + I_{\text{sleep}} \cdot T_{\text{sleep}}}{T_{\text{transmission}}} = 3.99 \mathrm{mA}
	\end{equation}
	
	and now we compute the battery life
	\begin{equation}
		L^B = \frac{C}{I_{avg}^B} = 626.68 \text{ hours} = 0.071 \text{ years}
	\end{equation}
	
	\subsection*{Question 6}
	
	In Class C, the device will listen continuously to the downlink channel (no time is spent on standby), which means 
	
	\begin{equation}
		T_{rx}^C = T = 7200 \mathrm{s}
	\end{equation}
	
	\subsection*{Question 7}
	
	We calculate the average current
	\begin{equation}
		I_{avg}^C = \frac{I_{tx} \cdot T_{tx} + I_{rx} \cdot T_{rx}^C + I_{\text{sleep}} \cdot T_{\text{sleep}}}{T_{\text{transmission}}} = 0.01\mathrm{A}
	\end{equation}
	
	and now we compute the battery life
	\begin{equation}
		L^C = \frac{C}{I_{avg}^C} = 248.8 \text{ hours} = 10.37 \text{ days}
	\end{equation}
	
	\section{Sigfox MAC Performance}
	
	\subsection*{Question 8}
	
	We add the expected arrival rate of each of the $N-1$ other devices. Each device sends $\lambda r$ packets with probability $\frac{\delta_f}{W}$.
	
	So the arrival rate of packets in the vulnerability window is
	
	\begin{equation}
		\lambda_v = (N-1)\lambda r \cdot \frac{\delta_f}{W}
	\end{equation}
	
	\subsection*{Question 9}
	
	The probability of at least one arrival is 
	
	\begin{align}
		P_{\text{arrival}} &= 1 - \mathbb{P}[k = 0] \\
						   &= 1 - e^{\lambda_v T} \\
						   &= 1 - e^{(N-1)\lambda r \frac{\delta_f}{W}}
	\end{align}
	
	\subsection*{Question 10}
	
	Since each retransmission is independant, the probability of failure is
	
	\begin{equation}
		P_{\text{failure}} = \left(1 - e^{\lambda_v T}\right)^r = \left(1 - e^{(N-1)\lambda r \frac{\delta_f}{W}}\right)^r
	\end{equation}
	
	\section{LoRa Coverage}
	
	\subsection*{Question 11}
	
	Given $N_0$ and $W$, we compute the noise power
	
	\begin{equation}
		N = N_0 + 10\log_{10}(W) = -174 + 10\log_{10}(125000) = -123\text{dBm}
	\end{equation}
	
	and now we compute the receiver sensitivity
	
	\begin{equation}
		P_{\text{min}} = N + SNR = -123\text{dBm} + (-20 \text{dB}) = -143\text{dBm}
	\end{equation}
	
	\subsection*{Question 12}
	
	The total gain is $G = 6 + 10\log_{10}(2) = 12.93\text{dBi}$.
	
	The total loss is $L = 3\text{dB} + 18\text{dB} = 21\text{dB}$.
	
	The shadowing margin $M_s$ can be calculated with
	
	\begin{align}
		P_{out} &= 1 - Q\left(\frac{P_{tx} + G - L - M_s}{\sigma}\right) \\
		\iff M_s &= (P_{tx} + G - L) + \sigma Q^{-1}(1 - P_{out})
	\end{align}
	
	which gives
	
	\begin{equation}
		M_s = 17.05\text{dB}
	\end{equation}
	
	\subsection*{Question 13}
	
	\subsubsection*{MAPL(Maximum Allowable Path Loss)}
	
	Let's compute the MAPL(\href{https://wireless-systems.ece.gatech.edu/6604/2019-lectures/week2.pdf}{source})
	
	\begin{equation}
		\text{MAPL} = P_{tx} + G - L - P_{\text{min}} = 151.11\text{dB}
	\end{equation}
	
	this means the signal can lose up to $151.11\text{dB}$ before it becomes too weak to detect.

	\subsubsection*{Cell range in urban and rural areas}
	
	We use the Hata model with
	
	\begin{equation}
		\text{MAPL} = A + B\log_{10}(d) + C
	\end{equation}
	
	For urban areas we have
	
	\begin{align}
		\begin{cases}
			A_{\text{urban}} &= 69.55 + 26.16\log_{10}(f_c) - 13.82\log_{10}(h_b) = 282.97 \\
			B_{\text{urban}} &= 44.9 - 6.55\log_{10}(h_b) = 35.22 \\
			C_{\text{urban}} &= 3.2(\log_{10}(11.75f_c))^2 - 4.97 = 315.58
		\end{cases}
	\end{align}
	
	which gives
	
	\begin{equation}
		d_{\text{urban}} = 10^{\frac{\text{MAPL} - A_{\text{urban}} + C_{\text{urban}}}{B_{\text{urban}}}} = 164.3\text{km}
	\end{equation}
	
	For rural areas we have
	
	\begin{align}
		\begin{cases}
			A_{\text{rural}} &= A_{\text{urban}} \\
			B_{\text{rural}} &= B_{\text{urban}} \\
			C_{\text{rural}} &= 4.78(\log_{10}(f_c))^2 - 18.33\log_{10}(fc) + 40.94 = 259
		\end{cases}
	\end{align}
	
	which gives
	
	\begin{equation}
		d_{\text{rural}} = 10^{\frac{\text{MAPL} - A_{\text{rural}} + C_{\text{rural}}}{B_{\text{rural}}}} = 4.07\text{km}
	\end{equation}

\end{document}